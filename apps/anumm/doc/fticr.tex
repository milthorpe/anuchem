\documentclass[10pt,conference,onecolumn]{IEEEtran}
%
% Use packaged depending on need
%
% \usepackage{graphicx}
% \usepackage{algorithmic}
% \usepackage{citesort}
% \usepackage{scs}
%
\usepackage{graphicx}
\usepackage{amssymb,amsmath}
\usepackage{wrapfig}

\usepackage[
pdfauthor={Josh Milthorpe},
pdftitle={Simulation of Fourier transform ion cyclotron resonance mass spectrometer},
pdfcreator={pdftex},
pdfkeywords={Fourier transform ion cyclotron resonance, mass spectrometry, Penning trap},
pdfpagemode={none},
pdfstartview={FitH},
bookmarks={false}
]{hyperref}

% correct bad hyphenation here
\hyphenation{net-works}

\begin{document}

\section{Simulation of ions in a Penning trap}

Ions are confined radially by a magnetic field $\mathbf{B}$, with an ideal quadrupolar trapping potential of $V_T$.
The potential and field near the centre of the trap is approximated as 
\begin{align}
\Phi_T(x,y,z) &= V_T (\gamma' - \frac{\alpha'}{2l^2}(x^2 + y^2 - 2z^2)) &
\mathbf{E} &= - \nabla \Phi_T(x,y,z) = \frac{\alpha'}{l^2}(-x\mathbf{i} -y\mathbf{j} +2z\mathbf{k})
\end{align}
where $\gamma' = 1/3$ and $\alpha' = 2.77373$ are geometric factors for the cubic trap, and $l$ is the edge length\cite{Guan1995}.
The force experienced by an ion in the trap is
\begin{equation}
\mathbf{F} = q(\mathbf{E} + \mathbf{v} \times \mathbf{B})
\end{equation}
Charges and mass are simulated in atomic units, lengths in nm and time in ns.



\subsection{Integration scheme}
The code uses the Boris integrator\cite{Boris1970} as formulated in Birdsall and Langdon\cite{Birdsall1985}.
This is a modified leapfrog scheme in which positions are calculated at times $..., n-1, n, n+1, ...$ and velocities at times $..., n^{-1/2}$, $n^{+1/2}, ...$.
\begin{align}
\mathbf{v}^- &= \mathbf{v}^{n-1/2} + \frac{q \mathbf{E}}{m} \frac{\Delta t}{2} \\
\mathbf{v}'  &= \mathbf{v}^- + \mathbf{v}^- \times \mathbf{t} & \mathbf{t}   &= \frac{q\mathbf{B}}{m} \frac{\Delta t}{2} \\
\mathbf{v}^+ &= \mathbf{v}^- + \mathbf{v}' \times \mathbf{s} & \mathbf{s}   &= \frac{2\mathbf{t}}{1 + t^2} \\
\mathbf{v}^{n+1/2} &= \mathbf{v}^+ + \frac{q \mathbf{E}}{m} \frac{\Delta t}{2}
\end{align}


\subsection{Ion motion}

The cyclotron angular velocity $\omega_c$ is predicted by $\omega_c = \frac{q B}{m}$, with cyclotron frequency $\nu_c = \frac{\omega_c}{2 \Pi}$\cite{Guan1995} (S.I. units - see Table \ref{tab:conversion}).  
To achieve 1\% phase error in simulated cyclotron frequency requires $\omega \Delta t \lesssim 0.3$\cite{Birdsall1985,Patacchini2009}.

Ions are excited to a uniform radius $r$, with different velocities and kinetic energies.
The magnitude of ion velocity in the $xy$ plane is predicted by $v_{xy} = \frac{qBr}{m}$ (S.I. units).

The electric field gives rise to a magnetron motion of a lower frequency $\omega_z$.
In a quadrupolar trapping potential of $V_T$, the modified cyclotron frequency $\omega_+$ is predicted by
\begin{align}
\omega_+ &= \frac{\omega_c}{2} + \sqrt{\frac{\omega_c^2}{4} - \frac{\omega_z^2}{2}} &
\omega_z &= \sqrt{\frac{2 \alpha q V_T}{m l^2}}
\end{align}
The modified cyclotron frequency is measured by peaks in the Fourier transform of the induced current time signal.

\subsection{Induced current}

In experiment, the current is measured between detector plates on opposite walls of the cube parallel to the magnetic field.

In the simulation, current is induced by the movement of the `image' $\mathbf{E}_{image}(\mathbf{r})$ associated with each ion, that is, the difference in the electric field generated by the ion at each of the two detector plates\cite{Guan1995}.
\begin{align}
I &= \sum_{i=1}^N q_i\mathbf{v}_i \cdot \mathbf{E}_{image}(\mathbf{r}_i) & \mathbf{E}_{image}(\mathbf{r}) &= - \frac{\beta'}{l} r_j & \beta' &= 0.72167
\end{align}


\subsection{Evaluation of electrostatic potential and forces}

In addition to the influence of the trapping field and the ion-image interaction, each ion experiences a repulsive Coulomb force from every other ion in the packet.
The calculation of these forces is nominally $O(N^2)$, which makes the simulation of large ion packets infeasible unless some approximation is used to reduce the computational complexity.

Ion excitation and detection is typically performed in a vacuum, or in an environment of low-pressure neutral gas particles.
To produce a detectable ICR signal, it is necessary to excite the ions so as to produce a highly coherent circular motion.
Thus the distribution of charged particles within the FT-ICR chamber is highly non-uniform.
Previous simulations have either used the particle-in-cell approximation\cite{Leach2009} or virtual particles to reduce the number of interactions to be simulated\cite{Fujiwara2010}.
However, particle-in-cell simulations are best suited to uniform particle distributions with a low required accuracy\cite{Greengard1989}.
In contrast, the adaptive fast multipole algorithm\cite{Cheng1999} provides guaranteed error bounds for non-uniform distributions.

\subsection{Fast multipole algorithm}
\label{sec:fmm}

\begin{wraptable}{r}{0.4\columnwidth}
\label{tab:tree_params}
\begin{tabular}{l|l|l}
\hline \hline
\textbf{N} & \textbf{s} & \textbf{leaf box size} (m) \\
\hline
$10^4$ & 3 & $1.95 \times 10^{-6}$ \\
$10^5$ & 4 & $2.44 \times 10^{-7}$ \\
$10^6$ & 5 & $3.10 \times 10^{-8}$ \\
\hline \hline
\end{tabular}
 \caption{Octree parameters, $N_0 = 100$}
\end{wraptable}

The cubic simulation space is divided into an octree of $s$ levels according to a desired mean number of particles per lowest level box $N_0$ such that $s = \left\lceil log_8(N / N_0) \right\rceil$.
The side length of a \emph{leaf} box at the lowest level is therefore $a / 8^s$.
Tree parameters are given for sample problem sizes where $N_0 = 100$ in table \ref{tab:tree_params}.



\subsection{Replication of previously published simulation results}

\begin{table}[htbp]
 \centering	  	  
 \caption{Replicating: Han and Shin\cite{Han1997}. Input, predicted and measured simulation parameters}
\label{tab:han}
\begin{tabular}{c|c|c|c|c|c|c|c|c|c|c}
 \hline \hline
  \multicolumn{11}{|c|}{$B = 0.7646 T, V_T = 1.0 V, l = 0.047, \Delta t = 25ns$} \\
 \hline \hline
 \multicolumn{4}{|c|}{ } & \multicolumn{2}{|c|}{\textbf{predicted}} & \textbf{measured} & \multicolumn{4}{|c|}{\textbf{error: timestep}} \\ 
 \hline
 \textbf{species} & \textbf{charge} & \textbf{mass} & \textbf{$r$ (mm)} & \textbf{$\nu_+$ (Hz)} & \textbf{$v_0$ (m/s)} & \textbf{$\nu_+'$ (Hz)}  & \textbf{$\Delta t$ (ns)} & \textbf{$\epsilon: \Delta t$}& \textbf{$\epsilon: \Delta t$ / 10} & \textbf{$\epsilon: \Delta t$ * 10}\\ 
 \hline
 $\mathsf{HCO^+}$ & 1 & 29.0182 & 5.0 & 404,356 & $1.27 \times 10^4$ & 404,220 & 118 & \\
 $\mathsf{CH_3CO^+}$ & 1 & 43.04462 & 5.0 & 272,508 & $8.57 \times 10^3$ & 272,460 & 175 & \\
 \hline \hline
\end{tabular}
\end{table}

\begin{table}[htbp]
 \centering	  	  
 \caption{Replicating: Leach et al.\cite{Leach2009}. Input, predicted and measured simulation parameters}
\label{tab:leach}
\begin{tabular}{c|c|c|c|c|c|c|c|c|c|c}
 \hline \hline
  \multicolumn{11}{|c|}{$B = 7.0 T, V_T = 1.0 V, l = 0.0508, \Delta t = 25ns$} \\
 \hline \hline
 \multicolumn{4}{|c|}{ } & \multicolumn{2}{|c|}{\textbf{predicted}} & \textbf{measured} & \multicolumn{4}{|c|}{\textbf{error: timestep}} \\ 
 \hline
 \textbf{species} & \textbf{charge} & \textbf{mass} & \textbf{$r$ (mm)} & \textbf{$\nu_+$ (Hz)}  & \textbf{$v_0$ (m/s)} & \textbf{$\nu_+'$ (Hz)}  &  \textbf{$\Delta t$ (ns)} & \textbf{$\epsilon: \Delta t$} & \textbf{$\epsilon: \Delta t$ / 10} & \textbf{$\epsilon: \Delta t$ * 10}\\ 
 \hline
 $\mathsf{Cs^+}$                   & 1 & 132.9054 & 6.0 & 808,767 & $2.70 \times 10^4$ & 807,680 & 59 & \\
 $\mathsf{Xx^+ (\frac{m}{q} = 150 )}$ & 1 & 150.0 & 6.0 & 716,594 & $3.05 \times 10^4$ & 715,840 & 67 & \\
 \hline \hline
\end{tabular}
\end{table}

\subsection{Comparison with experiment: amino acids}
\label{sec:amino}

We performed a scan of a mixture of lysine and glutamine on a Bruker Apex 4.7T FT-ICR mass spectrometer at the ANU.
The core of this machine is a Penning trap of 4.7T / 1.0V of side length 1cm.
We performed a molecular dynamics simulation for an equimolar mixture of N singly-charged ions in total.
The simulation parameters and results are in table \ref{tab:amino}.

\begin{table}[htbp]
 \centering	  	  
 \caption{Amino acids in ANU mass spectrometer. Input, predicted and measured simulation parameters}
\label{tab:amino}
\begin{tabular}{c|c|c|c|c|c|c|c|c|c|c}
 \hline \hline
  \multicolumn{11}{|c|}{$B = 4.7 T, V_T = 1.0 V, l = 0.01, \Delta t = 25ns$} \\
 \hline \hline
 \multicolumn{4}{|c|}{ } & \multicolumn{2}{|c|}{\textbf{predicted}} & \textbf{measured} & \multicolumn{4}{|c|}{\textbf{error: timestep}} \\ 
 \hline
 \textbf{species} & \textbf{charge} & \textbf{mass} & \textbf{$r$ (mm)} & \textbf{$v_0$ (m/s)} & \textbf{$\nu_+$ (Hz)} & \textbf{$\nu_+'$ (Hz)} & \textbf{$\Delta t$ (ns)} & \textbf{$\epsilon: \Delta t$}& \textbf{$\epsilon: \Delta t$ / 10} & \textbf{$\epsilon: \Delta t$ * 10}\\ 
 \hline
 $\mathsf{glutamine}$ & 1 & 147.07698 & 3.0 & $9.25 \times 10^3$ & 489,779 & 489,540 & 106 & \\
 $\mathsf{lysine}$ & 1 & 147.11336 & 3.0 & $9.25 \times 10^3$ & 489,658 & 489,407 & 106 & \\
 \hline \hline
\end{tabular}
\end{table}

\subsection{Peak coalescence}

Vladimirov et al.\cite{Vladimirov2011} give an expression for the minimum number of ions required for coalescence of two clouds of (singly-charged) ions of similar masses $m_1$ and $m_2$:
\begin{align}
N &= 4.87 \times 10^8 \frac{a^2 R B^2 (m_2 - m_1)}{m^2}
\end{align}

(For $a,R$ in mm, $m_1 ,m_2, m$ in Da, $B$ in Tesla.)

Applying this formula to the experiment described in section \ref{sec:amino}, where ion cloud major axis $a \approx 1$ mm, ion cyclotron radius $R \approx 3$ mm, and average mass $m = 147.09517$, we find the minimum number of ions required for coalescence is $\approx 55000$.

\begin{table}[htbp]
 \centering	  	  
 \caption{Conversion factors}
\label{tab:conversion}
\begin{tabular}{l|l|l}
\hline \hline
\textbf{property} & \textbf{unit} & \textbf{SI unit} \\
\hline
mass & amu & $1.660538921 \times 10^{ - 27} kg$ \\
charge & e   & $1.60217653 \times 10^{ - 19} C$ \\
\hline \hline
\end{tabular}
\end{table}

\bibliographystyle{IEEEtran}
\bibliography{fticr}

\end{document}

