\documentclass[10pt,conference,onecolumn]{IEEEtran}
%
% Use packaged depending on need
%
% \usepackage{graphicx}
% \usepackage{algorithmic}
% \usepackage{citesort}
% \usepackage{scs}
%
\usepackage{graphicx}
\usepackage{amssymb,amsmath}

\usepackage[
pdfauthor={Josh Milthorpe},
pdftitle={Simulation of Fourier transform ion cyclotron resonance mass spectrometer},
pdfcreator={pdftex},
pdfkeywords={Fourier transform ion cyclotron resonance, mass spectrometry, Penning trap},
pdfpagemode={none},
pdfstartview={FitH},
bookmarks={false}
]{hyperref}

% correct bad hyphenation here
\hyphenation{net-works}

\begin{document}

\section{Simulation of ions in a Penning trap}

Ions are confined radially by a magnetic field $\mathbf{B}$, with an ideal quadrupolar trapping potential of $V_T$.
The potential and field near the centre of the trap is approximated as 
\begin{align}
\Phi_T(x,y,z) &= V_T (\gamma' - \frac{\alpha'}{2l^2}(x^2 + y^2 - 2z^2)) &
\mathbf{E} &= - \nabla \Phi_T(x,y,z) = \frac{\alpha}{l^2}(-x\mathbf{i} -y\mathbf{j} +2z\mathbf{k})
\end{align}
where $\gamma' = 1/3$ and $\alpha' = 2.77373$ are geometric factors for the cubic trap, and $l$ is the edge length\cite{Guan1995}.
The force experienced by an ion in the trap is
\begin{equation}
\mathbf{F} = q(\mathbf{E} + \mathbf{v} \times \mathbf{B})
\end{equation}
Charges and mass are simulated in atomic units, lengths in nm and time in ns.



\subsection{Integration scheme}
The code uses the Boris integrator\cite{Boris1970} as formulated in Birdsall and Langdon\cite{Birdsall1985}.
This is a modified leapfrog scheme in which positions are calculated at times $..., n-1, n, n+1, ...$ and velocities at times $..., n^{-1/2}$, $n^{+1/2}, ...$.
\begin{align}
\mathbf{v}^- &= \mathbf{v}^{n-1/2} + \frac{q \mathbf{E}}{m} \frac{\Delta t}{2} \\
\mathbf{v}'  &= \mathbf{v}^- + \mathbf{v}^- \times \mathbf{t} & \mathbf{t}   &= \frac{q\mathbf{B}}{m} \frac{\Delta t}{2} \\
\mathbf{v}^+ &= \mathbf{v}^- + \mathbf{v}' \times \mathbf{s} & \mathbf{s}   &= \frac{2\mathbf{t}}{1 + t^2} \\
\mathbf{v}^{n+1/2} &= \mathbf{v}^+ + \frac{q \mathbf{E}}{m} \frac{\Delta t}{2}
\end{align}


\subsection{Ion motion}

The cyclotron frequency $\omega_c$ is predicted by $\omega_c = \frac{q B}{m}$\cite{Guan1995}.  
To achieve 1\% phase error in simulated cyclotron frequency requires $\Omega \Delta t \lesssim 0.3$\cite{Birdsall1985,Patacchini2009}.
The cyclotron radius $r$ is predicted by $r = \frac{m v}{|q| B}$ and is measured for a single particle simulation by the distance between maxima in the $x$ dimension.

In a quadrupolar trapping potential of $V_T$, the modified cyclotron frequency $\omega_+$ is predicted by
\begin{align}
\omega_+ & \equiv \frac{\omega_c}{2} + \sqrt{\frac{\omega_c^2}{4} - \frac{\omega_z^2}{2}} &
\omega_z &= \biggl(\frac{2 \alpha q V_T}{m l^2}\biggr)^{1/2}
\end{align}
The modified cyclotron frequency is measured by peaks in the Fourier transform of the induced current time signal.

\subsection{Induced current}

FTICR-MS measures the current induced between detector plates on walls of the cube parallel to the magnetic field.
In the simulation, current is induced by the movement of the `image' $\mathbf{E}_{image}(r)$ associated with each ion, that is, a charge equal in magnitude to the potential that would be produced at the ion position due to a unit potential applied to the detector electrode\cite{Guan1995}.
\begin{align}
I &= q\mathbf{v} \cdot \mathbf{E}_{image}(\mathbf{r}) & \mathbf{E}_{image} &= - \frac{\beta'}{l} r_j & \beta' &= 0.72167
\end{align}

\subsection{Simulation results}

\begin{table}[htbp]
 \centering	  	  
 \caption{Replicating: Han and Shin\cite{Han1997}. Input, predicted and measured simulation parameters}
\label{tab:han}
\begin{tabular}{c|c|c|c|c|c|c|c|c|c|c|c}
 \hline \hline
  \multicolumn{12}{|c|}{$B = 0.7646 T, V_T = 0.0 V, l = 0.047$ TODO rerun with $V_T = 1.0V$} \\
 \hline \hline
 \multicolumn{4}{|c|}{ } & \multicolumn{2}{|c|}{\textbf{predicted}} & \multicolumn{2}{|c|}{\textbf{measured}} & \multicolumn{4}{|c|}{\textbf{error: timestep}} \\ 
 \hline
 \textbf{species} & \textbf{charge} & \textbf{mass} & \textbf{$v_0$ (m/s)} & \textbf{$\omega_c / 2\Pi$ (Hz)} & \textbf{$r$ (nm)} & \textbf{$\omega_c' / 2\Pi$ (Hz)}  & \textbf{$r'$ (nm)} & \textbf{$\Delta t$ (ns)} & \textbf{$\epsilon: \Delta t$}& \textbf{$\epsilon: \Delta t$ / 10} & \textbf{$\epsilon: \Delta t$ * 10}\\ 
 \hline
 $\mathsf{HCO^+}$ & 1 & 29.0182 & 10 & 404,618 & 3,933 & 404,531 & 3,933 & 118 & \\
 $\mathsf{CH_3CO^+}$ & 1 & 43.04462 & 10 & 272,770 & 5,835 & 272,777 & 5,835 & 175 & \\
 \hline \hline
\end{tabular}
\end{table}

\begin{table}[htbp]
 \centering	  	  
 \caption{Replicating: Leach et al.\cite{Leach2009}. Input, predicted and measured simulation parameters}
\label{tab:leach}
\begin{tabular}{c|c|c|c|c|c|c|c|c|c|c|c}
 \hline \hline
  \multicolumn{12}{|c|}{$B = 7.0 T, V_T = 1.0 V, l = 0.0508$} \\
 \hline \hline
 \multicolumn{4}{|c|}{ } & \multicolumn{2}{|c|}{\textbf{predicted}} & \multicolumn{2}{|c|}{\textbf{measured}} & \multicolumn{4}{|c|}{\textbf{error: timestep}} \\ 
 \hline
 \textbf{species} & \textbf{charge} & \textbf{mass} & \textbf{$v_0$ (m/s)} & \textbf{$\omega_+ / 2\Pi$ (Hz)} & \textbf{$r$ (nm)} & \textbf{$\omega_+' / 2\Pi$ (Hz)}  & \textbf{$r'$ (nm)} & \textbf{$\Delta t$ (ns)} & \textbf{$\epsilon: \Delta t$}& \textbf{$\epsilon: \Delta t$ / 10} & \textbf{$\epsilon: \Delta t$ * 10}\\ 
 \hline
 $\mathsf{Cs^+}$ & 1 & 132.9 & 27 & 807,826 & 5,313 & 807,656 & 5,319 & 59 & \\
 $\mathsf{? (\frac{m}{q} = 150 )}$ & 1 & 150.0 & 27 & 715,654 & 5,996 & 715,781 & 5,997 & 67 & \\
 \hline \hline
\end{tabular}
\end{table}

\begin{table}[htbp]
 \centering	  	  
 \caption{Conversion factors}
\label{tab:conversion}
\begin{tabular}{l|l|l}
\hline \hline
\textbf{property} & \textbf{unit} & \textbf{SI unit} \\
\hline
mass & amu & $1.660538921 \times 10^{ - 27} kg$ \\
charge & e   & $1.60217653 \times 10^{ - 19} C$ \\
\hline \hline
\end{tabular}
\end{table}

\bibliographystyle{IEEEtran}
\bibliography{fticr}

\end{document}

